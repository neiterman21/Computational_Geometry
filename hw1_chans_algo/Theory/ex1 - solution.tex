\documentclass{article}
\usepackage[utf8]{inputenc}
\usepackage[english]{babel}
\usepackage{amsthm}
\usepackage{tikz}
\usepackage{geometry}
\usepackage{amsmath}
\usepackage{amssymb}


\usepackage{mathtools}
\DeclarePairedDelimiter\ceil{\lceil}{\rceil}
\DeclarePairedDelimiter\floor{\lfloor}{\rfloor}


\geometry{
 a4paper,
 total={170mm,257mm},
 left=20mm,
 top=20mm,
 }

\newtheorem{theorem}{Theorem}[section]
\newtheorem{corollary}{Corollary}[theorem]
\newtheorem{lemma}[theorem]{Lemma}
\newtheorem{definition}[theorem]{Definition}
\newtheorem{claim}[theorem]{Claim}

\newcommand{\cb}[1]{( #1 )}
\newcommand{\bcb}[1]{\Big( #1 \Big)}
\newcommand{\bbcb}[1]{\bigg( #1 \bigg)}

\newcommand{\myset}[1] {\{ #1 \}}
\newcommand{\bmyset}[1] {\Big\{ #1 \Big\}}
\newcommand{\bbmyset}[1] {\bigg\{ #1 \bigg\}}

\newcommand{\norm}[1]{\left\lVert#1\right\rVert}

\newcommand{\sqb}[1] {[ #1 ]}
\newcommand{\bsqb}[1] {\Big[ #1 \Big]}
\newcommand{\bbsqb}[1] {\bigg[ #1 \bigg]}

\newcommand{\abs}[1] {\mid#1\mid}
\newcommand{\angles}[1] {\langle #1 \rangle}
\newcommand{\QEDB}{\hfill{\square}}

\title{Computational Geometry EX1 - Solution}
\author{Abraham Khanukaev, Evgeny Naiterman-Hershkovitz}
\date{}

\begin{document}
\maketitle
\section{Question 1}

\begin{definition}
    The orientation function:
    \begin{equation*}
        orient((p_x, p_y), (q_x, q_y), (r_x, r_y)) = sign\left(
        \begin{vmatrix}
            1 & 1 & 1\\
            p_x & q_x & r_x\\
            p_y & q_y & r_y\\
        \end{vmatrix}
        \right)
        = sing((q_xr_y - r_xq_y) - (p_xr_y-r_xp_y) + (p_xq_y - q_xp_y))
    \end{equation*}
\end{definition}

\begin{claim}
All of the following holds:
\begin{itemize}
    \item $orient(p,q,r) = orient(q, r, p)$
    \item $orient(p,q,r) = orient(r, p, q)$
    \item $orient(p,q,r) = -orient(p, r, q)$
    \item $orient(p,q,r) = -orient(q, p, r)$
    \item $orient(p,q,r) = -orient(r, q, p)$
\end{itemize}
\end{claim}

\begin{proof} We prove each of the points.
    \begin{itemize}
    \item We reach it by taking the `minus' out of the second and the third terms.
    $$ (q_xr_y - r_xq_y) - (p_xr_y-r_xp_y) + (p_xq_y - q_xp_y) = (r_xp_y - p_xr_y) - (q_xp_y - p_xq_y) + (q_xr_y - r_xq_y) $$
    \item We reach it by taking the `minus' out of the first and the second terms.
    $$ (q_xr_y - r_xq_y) - (p_xr_y-r_xp_y) + (p_xq_y - q_xp_y) = (p_xq_y - q_xp_y) -(r_xq_y - q_xr_y) + (r_xp_y - p_xr_y) $$
    \item We observe that $orient(p, q, r) = -1 \cdot -1 \cdot orient(p, q, r)$ 
        \begin{equation*} 
            \begin{split}
                orient(p, q, r) & = (q_xr_y - r_xq_y) - (p_xr_y-r_xp_y) + (p_xq_y - q_xp_y) \\
                                & = -1 \cdot -1 \cdot ((q_xr_y - r_xq_y) - (p_xr_y-r_xp_y) + (p_xq_y - q_xp_y))\\
                                & = -1 \cdot (-(q_xr_y - r_xq_y) + (p_xr_y-r_xp_y) - (p_xq_y - q_xp_y))\\
                                & = -1 \cdot ((r_xq_y - q_xr_y) - (p_xq_y - q_xp_y) + (p_xr_y-r_xp_y)) \\
                                & = -1 \cdot orient(p, r, q) \\
            \end{split}
        \end{equation*}
    \item We observe that $orient(p, q, r) = -1 \cdot -1 \cdot orient(p, q, r)$ 
        \begin{equation*} 
            \begin{split}
                orient(p, q, r) & = (q_xr_y - r_xq_y) - (p_xr_y-r_xp_y) + (p_xq_y - q_xp_y) \\
                                & = -1 \cdot -1 \cdot ((q_xr_y - r_xq_y) - (p_xr_y-r_xp_y) + (p_xq_y - q_xp_y))\\
                                & = -1 \cdot (-(q_xr_y - r_xq_y) + (p_xr_y-r_xp_y) - (p_xq_y - q_xp_y))\\
                                & = -1 \cdot ((p_xr_y-r_xp_y) -(q_xr_y - r_xq_y) + (q_xp_y - p_xq_y)) \\
                                & = -1 \cdot orient(q, p, r) \\
            \end{split}
        \end{equation*}
    \item We observe that $orient(p, q, r) = -1 \cdot -1 \cdot orient(p, q, r)$ 
        \begin{equation*} 
            \begin{split}
                orient(p, q, r) & = (q_xr_y - r_xq_y) - (p_xr_y-r_xp_y) + (p_xq_y - q_xp_y) \\
                                & = -1 \cdot -1 \cdot ((q_xr_y - r_xq_y) - (p_xr_y-r_xp_y) + (p_xq_y - q_xp_y))\\
                                & = -1 \cdot (-(q_xr_y - r_xq_y) + (p_xr_y-r_xp_y) - (p_xq_y - q_xp_y))\\
                                & = -1 \cdot ((q_xp_y - p_xq_y) -(r_xp_y - p_xr_y) + (r_xq_y - q_xr_y)) \\
                                & = -1 \cdot orient(r, q, p) \\
            \end{split}
        \end{equation*}
    \end{itemize}
\end{proof}

\begin{claim}
    If $orient(p,x,q) = orient(q, x, r) = orient(r, x, p)$, then $orient(r, q, p)$
    is also equal to them. 
\end{claim}

\noindent Note. Before going further, for the ease of use we will defined that
\begin{equation*}
    pq = 
    \begin{vmatrix}
        p_x & q_x \\
        p_y & q_y \\
    \end{vmatrix} = p_xq_y - p_yq_x
\end{equation*}
and that $pq = -qp$. Therefore, by that definition we have that 
$$ orient(p, q, r) = sign(qr - pr + pq).$$ 

\begin{proof}
    \begin{equation*}
        \begin{split}
            orient(p, x, q) + orient(q, x, r) + orient(r,x,p) & = 
                sign(xq - pq + px) + sign(xr - qr + qx) + sign(xp - rp + rx) \\ 
                & \overbrace{=}^{(1)} sign(xq - pq + px + xr - qr + qx + xp - rp + rx) \\ 
                & \overbrace{=}^{(2)} sign(- pq - qr - rp) \\
                & \overbrace{=}^{(3)} sign(qp - rp + rq) \\
                & = orient(r, q, p)\\
        \end{split}
    \end{equation*}
Where:

\noindent (1) By assumption they have the same signe, therefore, we can take it out
without affecting the final result. 

\noindent (2) Algebra based on definition-observation above. For example, $xq + qx = 0$. 

\noindent (3) Algebra based on definition-observation above. For instace, $-pq = qp$.

\end{proof}

\begin{claim}
    If the conditions of previous claim hold, then $x$ is a convex combination 
    of $p$, $q$ and $r$.
\end{claim}

\begin{proof}
    Let us begin by recalling the Cramer's rule and some basic linear algebra concepts. 
    First, recall that the linear system of form $A\cdot x = b$, has solution if and only if 
    $det(A) \neq 0$. And, we want to find sulution for the following system: 

    \begin{equation*} 
        \overbrace{\left[\begin{matrix}
            r_x & q_x & p_x \\
            r_y & q_y & p_y
        \end{matrix}\right]}^{A} \cdot \overbrace{\left[
            \begin{matrix}
            \alpha_1 \\ \alpha_2 \\ \alpha_3
        \end{matrix}\right]}^{\alpha} = \left[ \begin{matrix} x_x \\ x_y \end{matrix} \right]
    \end{equation*} 
    By previous lemma, it follows that $det(A) \neq 0$, therefore, there exists 
    $\alpha \in \mathbb{R}^3$ such that $A\cdot \alpha = x$. All we left to do 
    is: (1) Find such $\alpha$, and (2) show that $\alpha_1 + \alpha_2 + \alpha_3 = 1$. 
    To do so, we use Cramer's rule. 
    \\
    \\
    \noindent \textit{Theorem (Cramer)}. Let $A \cdot \alpha = \beta$ be linear 
    system, where $A = [\beta_1, \beta_2, ..., \beta_n]$ such that $\beta_i \in 
    \mathbb{R}^k$, and $\beta \in \mathbb{R}^k$ as well. If $det(A) \neq 0$, then
    $\forall j \in [1...n]$ $\alpha_j = det(A_j)/deta(A)$. Where for each $j \in 
    [1...n]$ we have that $$A_j = [\beta_1, \beta_2, ..., \beta_{j-1}, \beta, 
    \beta_{j+1}, ... \beta_n],$$ we replace the $\beta_j$ vector with $\beta$. 
    \\
    \\
    \indent We use Cramer's theorem to conclude the following.
    $$ 
    \alpha_1 = \frac{qr - xp + xq}{pq - rp + rq}, \quad \alpha_2 = \frac{xp - rp + rx}{pq - rp + rq}, 
    \quad \alpha_3 = \frac{qx - rx + rp}{pq - rp + rq},
    $$ 
    and from the previous lemma we conclude that $\alpha_1 + \alpha_2 + \alpha_3 = 1$ as required. 

\end{proof}

\begin{claim}
If $orient(p, x, q) = orient(p, x, r) = orient(p, x, s) = orient(q, x, r) = 
orient(r, x, s)$, then $orient(q, x, s)$ is also equal to them.
\end{claim}

\end{document}